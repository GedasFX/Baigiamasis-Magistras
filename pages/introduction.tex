\section{Introduction}

\subsection{Motivation}

With the emergence of COVID-19, work from home has rapidly grown in popularity. It has been especially noticeable in the IT industry. This phenomenon has led some businesses to transition to operate fully remote \cite{ozimek2020future}, allowing for potential customers, clients, and employees to operate with the companies' IT systems from all around the globe.

Identity verification is a significant roadblock when establishing a remote work policy. In some managerial businesses, such as logistics, it is essential to assure the authenticity of persons signing in to perform their duties. This security requirement is essential for those dealing with contracts, where one input can cost thousands of euros in losses. Traditionally, as work was always on-premises, it was easy to verify the identity with the help of an identity document. With the constraints of operations being fully remote, companies no longer have the luxury to perform such a check.

Establishing identity online for potential employees and clients is not the only use case for digital identity. Organizations such as the British Council employ privacy undermining practices. As part of registration to the IELTS exam, they require their customers to submit a photocopy of their identity document for verification purposes \cite{ielts-howtoregister}. This process is a significant privacy concern since anyone could replicate the uploaded document. Having no agency over their documents is of great concern for the end-users, making them reluctant to use the company services. Replacing the document upload with a digital signature check is a more secure and trustworthy way of performing business.

After the EU introduced the eIDAS regulation, an alternative method for identity verification became available. All EU member states are mandated to implement an eID solution in their country and recognize other countries' eID solutions \cite{eulaw-eidas}. Each eID solution guarantees some degree of authenticity, from substantial to high, allowing for verifying a persons' identity with trustworthy means.

Particular risks exist that businesses must be aware of before integrating an eID authentication service. There are no comprehensive resources outlining the obstacles and costs of implementing eID authentication in the private sector. Unknown risks are an excellent deterrent for innovation and make companies reluctant to use new technologies. Proper research into this subject may lead companies to take risks associated with implementing new technology and kickstart the mainstream adoption of eIDs in the private sector. 

\subsection{Research Problem}

The main goal of the thesis is to investigate if the advantages provided by eIDs are sufficient enough to outweigh the risks associated with adopting new technology in the free market and shine a light on the costs associated with implementation. From this goal, the extracted research question is as follows: 

\textbf{What is the best eID authentication option available for an Estonian EU targeting enterprise for use in their Web-based Single Sign-On (SSO)?}
\todo{Maybe it would be good to change to "is the market ready"}

The research question can be refined further into additional sub-questions:

\begin{itemize}
    \item What advantages do eIDs provide?
    \item How should an authentication scheme change to support eID?
    \item What privacy considerations must companies take when processing user data?
    \item What are the different eID authentication providers available to Estonia's private sector?
    \begin{itemize}
        \item What risks does the eID provider transfer away from the consumer company?
        \item What is the market reach (in countries) of a given solution?
        \item Where are the weak points in the protocol used? How should a company assess them?
        \item How expensive is the solution to operate?
    \end{itemize}
\end{itemize}

With these questions answered, it should be possible for any company to understand the risks and benefits of any given solution eID solution.

\subsection{Scope and goal}

There will be some assumptions about the company wishing to implement eID authentication in the thesis.

\paragraph{Company already uses an HTTP-based SSO (in the cloud or on-premises)} Non-HTTP-based SSOs or other esoteric authentication systems will not be considered.

\paragraph{Company is committed to getting some form of eID authentication system in place} This means market research was done, and the management found that it is favorable to invest in eID authentication, as the users would adopt this authentication mechanism.

\paragraph{The eID provider must be accessible by an Estonian company} Other countries may also provide eID solutions. However, for the scope of the thesis, only solutions available to Estonian citizens and companies will be considered.

% ###### Preparation:

% 1. Create a list of all commercially available eID providers;

%    > [Literature review] There is a list of services using Estonian eID solutions by the UT. Some entries in that list are eID providers themselves. Task is to filter out all providers.

% 2. Create an SSO Auth server;

%    > This task is the preparation task for the actual research. The server allows users to log in with username/password, and should be able to support federated login identity providers.
%    >
%    > For the purpose of this research, Microsoft Azure AD B2C will be used.

% 3. Create a dummy Single Page Application (SPA) website;

%    > The website can be used as a client to the SSO Auth server. In the website there should be a button to sign in with username/password. There also needs to be another section for signing in with eID (does nothing at this stage).
%    >
%    > For the purpose of this research, a Next.JS React SPA, deployed at Vercel will be used.

% ###### Analysis (for each of the eID providers):

% 1. Perform market reach of the eID provider;

%    > [Literature review] Calculate the amount of market gained by implementing the authentication method. Measure in countries/population. 

% 2. Analyze the validity of signature;

%    > [Literature review] Investigate the signature authenticity guarantees, and what obligations/measures provider has taken to provide the service. In the case of root trust providers, they are under high levels of scrutiny, required by eIDAS. In case of forwarding, they may have insurance, certifications. Specify them.

% 3. Investigate the contractual requirements;

%    > Most eID service providers require payment for operational use of their service. Investigate and note all monetary costs.

% 4. Integrate an eID provider into SSO server;

% > Measure the complexity of development. Write a summary, no more than 1 paragraph in length describing noteworthy oddities and challenges faced in implementing. Provide a short summary of practical implementation steps.
% >
% > If there will be extra maintenance challenges discovered also note them down. It is now assumed to be that maintenance costs are similar and per system basis (implementing 2 providers would double the relative cost).

% ###### Results:

% Compare options:

% > Comparison will be done on the following variables: 
% >
% > 1. Market reach (in countries)
% > 2. Trust level
% > 3. Operational cost (fixed, variable)
% > 4. Implementation complexity

\subsection{Contribution}

The thesis aims to fill the research gap for the use of eID in the private sector. There is some research about connecting eIDAS nodes. However, the focus was on connecting eIDAS nodes of other countries and not connecting customers to the eIDAS infrastructure, which is mainly off-limits.

The thesis contains the following contributions:

\begin{enumerate}
    \item enumeration of eID service providers in Estonia;
    \item analysis of personal data storage under GDPR for use on authentication;
    \item comparison of the different flavors of eID service providers;
    \item assessment of various data transfer protocols eID service providers use;
    \item display of example on how a company could implement an eID service into an SSO;
    \item disclose a vulnerability in Dokobit's identity gateway;
    \item feedback on the usability complexity for eeID project;
\end{enumerate}


% #### Research methods

% The research method would be exploratory. The idea is to compare different options, therefore discovery of options, and comparison is required. Each option will be measured by market reach, trust level, operational cost (fixed, variable), and implementation complexity. Market reach, trust level, and operation costs are part of the discovery process and are answerable by reading trough the literature. The implementation complexity analysis will use a model to assign a complexity to the documentation and the implementation process, and compare received value with other eID providers. No recommendations will be made with respect to if it is worth implementing an eID solution, as the context will be different from business to business, however some objective conclusions could still be brought out out of the comparison.

% Validation process would be reproducing the steps outlined in the Scope chapter. Most of the comparison points are publicly available, and the complexity analysis would need to follow the same steps, as outlined in the framework.

\paragraph{Structure of work}
The document will consist of the following main chapters:

\TODO{ISO standard to pick trust level for if companies even need it}
\TODO{What are different options in eIDAS, e.g. what is a QESSD}
\TODO{What are the differences between primary services and middlemen, advantages disadvantages}
\TODO{What are the weakpoints in the company structrue}
\TODO{What is the research model?}
\TODO{Findings about ID Card, Dokobit, and eeID}
\TODO{Non web-based SSO?}
\TODO{Ponder about the advantages of middleware pseudominization? Say instead of personal code you get some arbitrary ID that matters only in the system}