\section{Introduction}

\subsection{Motivation}

% #### Motivation

\paragraph{}

With the emergence of COVID-19, work from home has rapidly grown in popularity. It has been especially noticeable in the IT industry. This phenomenon has led some businesses to transition to operate fully remote \cite{ozimek2020future}, allowing for potential customers, clients, and employees to operate with the companies' IT systems from all around the globe.

Identity verification is a significant roadblock when establishing a remote work policy. In some managerial businesses, such as logistics, it is essential to assure the authenticity of persons signing in to perform their duties. Traditionally, as work was always on-premises, it was easy to verify the identity with the help of an ID Card or equivalent and physical verification. With the constraints of operations being fully remote, companies can no longer perform such a check.

Establishing identity online for potential employees and clients is not the only use case for digital identity. Organizations such as the British Council employ privacy undermining practices. They require their customers to submit a photocopy of their identity document for verification purposes \cite{ielts-howtoregister}. This process is a significant privacy concern since anyone could replicate the uploaded document. Having no agency over their documents is of great concern for the end-users, and they would be reluctant to use the company services. Replacing the document upload with a digital signature check is a more secure and trustworthy way of performing business.

After the EU introduced the eIDAS regulation, an alternative method for identity verification became available. All EU member states are mandated to implement an eID solution in their country and recognize other countries' eID solutions \cite{eulaw-eidas}. Each eID solution comes with an identity certificate and means to prove it by signing a challenge via public-key cryptography. Because of this regulation, it is now possible to obtain a persons' legal identity with trustworthy means.

Particular risks exist that businesses must be aware of before integrating an eID authentication service. There are no comprehensive resources outlining the obstacles and costs associated with implementing eID in the private sector. Lack of information makes it difficult to assess risks and estimate the resources required [?\todo{Cite}]. Unknown risks are an excellent deterrent for innovation and make companies reluctant to use new technologies. Proper research into this subject may lead companies to take risks associated with the implementation and kickstart the mainstream adoption of eIDs in the private sector. 

% #### Research Problem

% Convincing investors requires to conduct viability analysis, and analyze different options. The comparison of options can be the basis of my research question:

% > What are the differences between different eID authentication options available to a large Estonia based, EU market targeting company, wishing to implement it as an identity provider in their Web-based Single Sign-On (SSO)?

% We can dissect the more ambiguous parts of the question into additional sub-questions:

% > **differences between**
% > What is the criteria (properties/aspects) for comparing different options?
% >
% > **different eID authentication options**
% > What eID authentication options are available to the Estonia's private sector?
% >
% > **EU market targeting**
% > What is the market reach (in countries) of implementing a given solution?
% >
% > **to implement [eID authentication]**
% > What are the operational, contractual, development, and maintenance costs or considerations associated to implementing and using an eID solution?

% Additionally, there is a small security/liability concern regarding validity of signatures, and that could influence the decision making process of a business:

% > What are the security, liability guarantees provided by each of the eID providers?

% #### Scope and goal

% ###### Preparation:

% 1. Create a list of all commercially available eID providers;

%    > [Literature review] There is a list of services using Estonian eID solutions by the UT. Some entries in that list are eID providers themselves. Task is to filter out all providers.

% 2. Create an SSO Auth server;

%    > This task is the preparation task for the actual research. The server allows users to log in with username/password, and should be able to support federated login identity providers.
%    >
%    > For the purpose of this research, Microsoft Azure AD B2C will be used.

% 3. Create a dummy Single Page Application (SPA) website;

%    > The website can be used as a client to the SSO Auth server. In the website there should be a button to sign in with username/password. There also needs to be another section for signing in with eID (does nothing at this stage).
%    >
%    > For the purpose of this research, a Next.JS React SPA, deployed at Vercel will be used.

% ###### Analysis (for each of the eID providers):

% 1. Perform market reach of the eID provider;

%    > [Literature review] Calculate the amount of market gained by implementing the authentication method. Measure in countries/population. 

% 2. Analyze the validity of signature;

%    > [Literature review] Investigate the signature authenticity guarantees, and what obligations/measures provider has taken to provide the service. In the case of root trust providers, they are under high levels of scrutiny, required by eIDAS. In case of forwarding, they may have insurance, certifications. Specify them.

% 3. Investigate the contractual requirements;

%    > Most eID service providers require payment for operational use of their service. Investigate and note all monetary costs.

% 4. Integrate an eID provider into SSO server;

% > Measure the complexity of development. Write a summary, no more than 1 paragraph in length describing noteworthy oddities and challenges faced in implementing. Provide a short summary of practical implementation steps.
% >
% > If there will be extra maintenance challenges discovered also note them down. It is now assumed to be that maintenance costs are similar and per system basis (implementing 2 providers would double the relative cost).

% ###### Results:

% Compare options:

% > Comparison will be done on the following variables: 
% >
% > 1. Market reach (in countries)
% > 2. Trust level
% > 3. Operational cost (fixed, variable)
% > 4. Implementation complexity

% #### Literature review

% Most of the contextual questions can be answered from literature review. Due to the practical nature of the research, 

% ##### eIDAS

% The eIDAS regulation [3] provided the groundwork for recognizing the signatures issued by other EU countries, by imposing strict security and privacy requirements. A concept of a Trust Service Provider (TSP) is established, which means every entity with that status, can be legally considered as a source of truth. Each member state maintains a list of TSP, with entities being trusted for certain tasks, such as timestamping, issuing certificates, or performing other trust activities. The regulation also requires member states to establish eID systems, if they haven't already, and make them able to be integrated in a federal system.

% The regulation was the basis for creating the eIDAS network. At its core, there are nodes [4]. These nodes connect to one another across countries, allowing users to authenticate with the eID of their home (issuer) country in host (current residence) country. This is done by redirecting the authentication request to the appropriate country, federating the identification process. This is useful, as one service provider would be able to open up the entirety of EU market. The main issue with using eIDAS nodes as an authentication method, is the restricted access to it. In email correspondence I learnt that in Estonia, access to this service is limited to public sector only, with plans to open it up to private sector in 2022.

% ##### Attempts of eIDAS implementations in private sector

% In academic literature, there are only two well documented cases of how the private sector would access the eIDAS node network.

% ###### eID@Cloud

% The project eiD@cloud [5], conducted May 2017 to September 2018, has discovered certain issues when attempting to connect to the infrastructure. It has discovered that there's still some differences between the national schemes and the integrations of said national schemes in a unique and interoperable net that must be the eIDAS in the context of the EU, and the deployment of each eIDAS node of each member state by the national politics go at different speeds, which create mistakes and lack of availability to complete the eIDAS project. The authors are pessimistic about the prospect that fully connected Europe can be achieved soon.

% ###### LEPS

% LEPS [6], conducted September 2017 to November 2018 has tried to achieve similar goals to eID@Cloud - to identify gaps in the eIDAS infrastructure. The main challenge identified, much like in the previous research, is the lack of Service Providers, the private sector could use to interface with the eIDAS network.

% ##### eID providers in Estonia

% Applied Cyber Security Group of University of Tartu maintains a list of e-services [7] which uses at least one eID authentication method in Estonia. The following authentication methods were listed: Bank Link, ID-card, Mobile-ID, Smart-ID, TARA, and HarID. 

% ###### Bank Link

% This method of authentication is primarily created for e-services to provide close integration with banks for easier payments. Additionally, it provides an additional method of authentication to those e-services [8]. In a thesis conducted in 2012 [9], it was discovered that "Internet bank authentication [is] extremely insecure". In the years after the publication, the security got significantly better, however even then, from March of 2021 RIA no longer supports authentications with TARA, due to lack of security assessments with regards to eIDAS [https://cybersec.ee/2021/02/22/cyber-security-newsletter-2021-02-22/].

% Due to lack of security scrutiny required to satisfy eIDAS, and a poor market reach, this authentication method will not be considered  purposes of this thesis.

% ###### ID-card

% Id cards are by far the most popular way to access their eID in Estonia, which is primarily due to the legal requirement of having one. Chapter 2 of the Identity Documents Act [10] requires all EU citizens residing in Estonia to hold an ID-card, with which they could access public services online. Because of this quirk, there are more active ID-cards issued, than there are people in Estonia [11].

% There are no variable costs to allow persons to log in to websites with their ID-card, as there are no per-transaction costs for ID card authentication as the certificate validity service (OCSP) can be queried for free [12]. 

% Based on the countries internal policies, the chips on ID-cards can have different data layouts, as it is not standardized. This means that there must be a specialized piece of software to read different countries' ID-cards, requiring additional development costs.

% Certificates to be installed in ID-cards are issued by SK ID Solutions which is a trust service provider for Qualified Certificates for e-signatures [13]

% ###### Mobile-ID

% 5 years after ID cards were started to be used in Estonia, SK ID Solutions, developed a mobile phone friendly way to access the users' eID for use in Estonia, and Lithuania [14]. This was done by extending the functionality of SIM cards to make them mimic functionality of ID-cards.

% The price of using Mobile-ID for the service provider varies based on usage, staring from 10 euro per month (10ct per request), to costing over 5 000 euro, where the effective cost is under 1ct for request [15]. For the end user, mobile operators can charge extra for their monthly subscription fee, based on the contract they have with them.

% Implementation of Mobile-ID would allow service providers to access the markets of two countries: Estonia, and Lithuania, as the technical implementation is identical.

% Investigating the poor market size of users using Mobile-ID, implementation analysis will be left outside of scope of the thesis.

% Certificates to be installed in SIM cards capable of using Mobile-ID are issued by SK ID Solutions which is a trust service provider for Qualified Certificates for e-signatures [13].

% ###### Smart-ID

% Smart-ID is the latest, and fastest growing source of eID, working in all 3 of the Baltic States [16]. It utilizes mobile phones as authentication, however unlike Mobile-ID, it does not require external hardware, and everything is handled in a combination of on-server and on mobile phone. Despite that, it is still eIDAS compliant, and was recognized as a QSCD, allowing it to create QES in 2018 [17].

% The price of using Smart-ID for service providers, much like Mobile-ID varies based on usage, staring from 50 euro per month (10ct per request), to costing over 20 000 euro, where the effective cost is under 1ct for request, based on the total amount of transactions performed within a month [18]. For users, unlike Mobile-ID, there are no telecommunication operators involved and there are no costs associated to using Smart-ID.

% Implementation of Smart-ID would allow users to access the markets of three countries: Estonia, Latvia, and Lithuania.

% Part of the certificate used by Smart-ID is stored by SK ID Solutions which is a trust service provider for Qualified Certificates for e-signatures [13]. The other part is stored on the device of the user.

% ###### TARA

% TARA is the Estonia's eIDAS node interface [19]. It provides the ability for users to sign in with the 3 primary sources of identity in Estonia, and with the eID schemes of other EU member states. According to the business description [20], its use is intended for governmental agencies only, and would not work in a private setting.

% ###### eeID service

% This is a service, that uses TARA as its trust anchor for authentication, allowing users to authenticate with all of the methods provided by TARA [21]. The main difference, is that this method is targeting the private sector, enabling companies to access entirety of EU eID market.

% The service is new and does not have pricing tiers, and currently sits at 9ct per request, regardless of the amount of requests [22].

% Using eeID service would allow users to access the markets of all eIDAS countries, which will eventually include all of EU and EEA. Right now the list of countries (14) are as follows: Estonia, Germany, Italy, Spain, Belgium, Luxembourg, Croatia, Portugal, Latvia, Lithuania, Netherlands, Czech Republic, Slovakia, and Denmark.

% ###### HarID

% This service was created for the youth of Estonia to access different educational institutions across Estonia [23]. ID-cards are only legally required to be held by citizens over age of 15, so everyone under, would have unable to access their school system. HarID accepts TARA sign in methods, and username & password. This authentication method is not accessible by non-education sector, and will be skipped for the purposes of this thesis.

% ###### Dokobit

% In the initial list of services using eID in Estonia, one service stood out - Dokobit [24]. They provide resources similar to eeID, in the way that they aggregate different eID methods (ID-card, Mobile-ID, and Smart-ID), as well as eID methods from other countries. The main difference between it, and eeID, is that authentication achieved by using the native implementations of the publicly available resources of each country, rather than relying on the eIDAS infrastructure.

% Pricing for Dokobit varies drastically, and the provided prices for the Baltic States [25] starts at 50 euro per month (7.1ct per request), going down to 4.2ct per request at 500 euro per month.

% Dokobit supports 11 countries: Estonia, Italy, Spain, Belgium, Latvia, Lithuania, Finland, Norway, Iceland, Poland, and Portugal [24].

% UAB Dokobit is a trust service provider for Qualified validation of qualified e-signature. It means that the service itself does not provide certificates, but validation of signatures is considered to be trustworthy under eIDAS.

% ##### Research methodology

% ###### Development complexity

% One of primary outcomes of the research is to measure the complexity of the development. A model [26] will be used to measure the complexity of the examples and documentation provided by the services, and assign that to measurable values, which can be compared.

% #### Novelty

% * The two researches done about the private sector focused on only the eIDAS implementation. 
% * Development cost analysis is ignored. 
% * No instructions as to how to properly connect to eIDAS node for businesses. 
% * No research has been done on the development costs on any of the Estonia's eID authentication methods.

% This thesis aims to fill the gaps by providing implementation instructions and comparison of 4 eID providers: Estonian ID-card, Smart-ID, Dokobit, and eeID.

% #### Research methods

% The research method would be exploratory. The idea is to compare different options, therefore discovery of options, and comparison is required. Each option will be measured by market reach, trust level, operational cost (fixed, variable), and implementation complexity. Market reach, trust level, and operation costs are part of the discovery process and are answerable by reading trough the literature. The implementation complexity analysis will use a model to assign a complexity to the documentation and the implementation process, and compare received value with other eID providers. No recommendations will be made with respect to if it is worth implementing an eID solution, as the context will be different from business to business, however some objective conclusions could still be brought out out of the comparison.

% Validation process would be reproducing the steps outlined in the Scope chapter. Most of the comparison points are publicly available, and the complexity analysis would need to follow the same steps, as outlined in the framework.