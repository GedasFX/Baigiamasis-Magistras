\section{Introduction}

\subsection{Motivation}

With the emergence of COVID-19, work from home has rapidly grown in popularity. It has been especially noticeable in the IT industry. This phenomenon has led some businesses to transition to operate fully remote \cite{ozimek2020future}, allowing for potential customers, clients, and employees to operate with the companies' IT systems from all around the globe.

Identity verification is a significant roadblock when establishing a remote work policy. In some managerial businesses, such as logistics, it is essential to assure the authenticity of persons logging in to perform their duties. This security requirement is essential for those dealing with contracts, where one input can cost thousands. Traditionally, as work was always on-premises, it was easy to verify the identity with the help of an identity document. With the constraints imposed by fully remote operations, companies no longer have the luxury to perform such a check.

Establishing identity online for potential employees and clients is not the only use case for digital identity. Organizations such as the British Council employ privacy undermining practices. As part of the registration process for the IELTS exam, they require their customers to submit a photocopy of their identity document for verification purposes \cite{ielts-howtoregister}. This process is a significant privacy concern since anyone could replicate the uploaded document. Having no agency over their documents is of great concern for the end-users, making them reluctant to use the company services. Replacing the document upload with a digital signature check is more secure and less privacy undermining way of performing business.

After the EU introduced the eIDAS regulation, an alternative method for identity verification became available \cite{eulaw-eidas}. All EU member states are mandated to implement an eID solution in their country and recognize other countries' eID solutions. Each eID solution guarantees some degree of authenticity, from substantial to high, allowing for verification of a persons' identity via trustworthy means.

Particular risks exist that businesses must be aware of before integrating an eID authentication service. There are no comprehensive resources outlining the obstacles and costs of implementing eID authentication in the private sector at this point in time. Unknown risks are an excellent deterrent to innovation, making companies reluctant to use new technologies. Proper research into this subject may lead companies to take risks associated with implementing new technology and kickstart the mainstream adoption of eIDs in the private sector.

\subsection{Research Problem}

The main goal of the thesis is to investigate what options companies have if they wish to integrate eID authentication into their day-to-day businesses and the steps they would need to take to adopt the technology. From this goal, the extracted research question is as follows:

\textbf{What is the best eID authentication solution available to an Estonian EU targeting enterprise?}

To help answer this question, we would need to refine it into additional sub-questions:

\begin{enumerate}
    \item What are the prerequisites for a given architecture to be able to support eID solutions?
    \item Are there any legal considerations companies must be aware of before integration?
    \item What are the different eID authentication solutions available to Estonia's private sector?
    \item How do various eID providers compare based on the following questions:
          \begin{enumerate}
              \item How trustworthy is it to process sensitive information?
              \item How large is its market reach (in countries)?
              \item How expensive is it to operate?
              \item Does it inconvenience the end users any more than the regular eID providers would?
              \item How complicated is it to integrate and maintain?
              \item Is the authentication protocol protected against common protocol attacks?
          \end{enumerate}
\end{enumerate}

From the initial question, the word "best" is ambiguous. This part was further narrowed down into six questions (4a-f) to help clear it up. These criteria were chosen based on the feedback provided by a CTO of a logistics company. We provide the full interview in the appendix \todo{Appendix no.}.

\subsection{Research Methods}

In the previous section, we outlined the questions we aim to answer in this thesis. In this section we will describe how we will obtain the data necessary to answer the questions.

\paragraph{Question 1: What are the prerequisites for a given architecture to be able to support eID solutions?}



\subsection{Scope}

To not cover every possible scenario, we will be making a couple of assumptions about the company wishing to implement eID authentication in the thesis.

\paragraph{Company already uses an HTTP-based SSO (in the cloud or on-premises)} When analyzing an eID solution for integration complexity, we will only consider using an HTTP-based SSO. We chose to support only a single local identity provider to eliminate as many variables as possible, as all analyzed solutions would have to fit into a similar structure.

\paragraph{Company is committed to getting some form of eID authentication system in place} This means they did the market research, and management found it favorable to invest in eID authentication. Analyzing which companies would benefit from eID authentication or if they should invest in the first place is outside the thesis's scope.

\paragraph{The eID provider must be accessible by an Estonian company} Other countries also provide eID solutions. However, for the scope of the thesis, only solutions originating from or heavily invested in Estonia will be considered.


\subsection{Contribution}

The thesis aims to fill the research gap on the use of eID in the private sector and provide a framework for researchers or people in managerial positions to compare different eID authentication providers.

The thesis contains the following contributions:

\begin{enumerate}
    \item enumeration of eID service providers in Estonia;
    \item analysis of personal data storage under GDPR for use on authentication;
    \item comparison of the different approaches eID service providers can take for integrating cross-border authentication;
    \item assessment of various data transfer protocols in use for eID authentication;
    \item display of example on how a company could integrate an eID service into an SSO;
    \item security assessment and disclosure of weaknesses in analyzed eID authentication providers;
\end{enumerate}

% #### Research methods

% The research method would be exploratory. The idea is to compare different options, therefore discovery of options, and comparison is required. Each option will be measured by market reach, trust level, operational cost (fixed, variable), and implementation complexity. Market reach, trust level, and operation costs are part of the discovery process and are answerable by reading trough the literature. The implementation complexity analysis will use a model to assign a complexity to the documentation and the implementation process, and compare received value with other eID providers. No recommendations will be made with respect to if it is worth implementing an eID solution, as the context will be different from business to business, however some objective conclusions could still be brought out out of the comparison.

% Validation process would be reproducing the steps outlined in the Scope chapter. Most of the comparison points are publicly available, and the complexity analysis would need to follow the same steps, as outlined in the framework.

\subsection{Structure of work}

The thesis will consist of the following main chapters:

\paragraph{Section 2: Background} This chapter contains literature relevant to understanding the terminology and concepts used later in the thesis. Additionally, it contains a list of currently available eID providers in Estonia.
\paragraph{Section 3: Related Work} This chapter covers similar topics covered by the thesis. These studies may be done with different technologies or in different countries.
\paragraph{Section 4: Architecture Definition} When integrating an eID provider into an existing system, one must first know how the system is composed. Here we provide a schema and process overview and address inherent weaknesses in the base system.
\paragraph{Section 5-7: Case Studies} These sections look at data flow, trust, pricing, security requirements, integration specifics, and discovered weaknesses for each provider in the thesis (eeID, Dokobit, Web eID). These sections also spotlight the advantages and disadvantages of each provider.
\paragraph{Section 8: Discussion} This section discusses other factors that would affect companies' decisions to integrate eID providers and attempts to answer the underlying question of which eID provider (if any) a company should integrate. This section was done in part with the help of a CTO of a logistics company.