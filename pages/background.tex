\section{Background}

% #### Literature review

% Most of the contextual questions can be answered from literature review. Due to the practical nature of the research, 

% ##### eIDAS

\subsection{eID}

In Estonia, digital identity has been around for over 20 years \cite{eelaw-idcard}. The Estonian government has loaded all identity cards issued with certificates enabling cardholders to identify themselves digitally. Compare the speed of adoption to Romania, where the first easy access to eIDs came in the form of new chip ID cards in August of 2021 \cite{romania-adopts-eid}.

Estonia's early adoption of eID, the political focus on digital government, has led to over 89\% of internet users accessing the e-government, landing it the first place in the EU \cite{eu-desi}. The 20 years of easy access to an eID has led to a stark difference to Romania, where only 16\% of internet users access the government services online.

Depending on the country a company would like to access the market, eID sign-in may confuse the potential clients. Early adopters must be aware of the widespread adoption of the eID infrastructure.

In different countries, the eID solution may vary wildly. There can also be more than one eID solution in a singular county.

\subsubsection{Impact }



\subsection{eIDAS}

The eIDAS regulation \cite{eulaw-eidas} provided the groundwork for recognizing the signatures issued by other EU countries by imposing strict liability and mutual-recognition requirements. The regulation introduced the concept of a Trust Service Provider (TSP), which allowed relying parties to have a trust anchor. Each member state maintains a list of TSPs, where each TSP is certified to perform specific tasks, such as timestamping or issuing signing certificates. The regulation also requires member states to establish eID systems, if they haven't already, and make them able to be integrated into a federal system.

The regulation was the basis for creating the eIDAS node network \cite{carretero2018federated}. These nodes connect across country borders, allowing users to authenticate with the eID of their home (eID issuer) country in the host (current residence) country. The eIDAS authentication protocol redirects the authentication requests to the appropriate country, federating the identification process. For the institutions trying to target the EU market, this provides a significant advantage since access to one node would mean access to all nodes in the EU.

The main issue private companies will encounter is the highly restricted access to any nodes. The eIDAS network is only concerned about connecting countries. To allow access to the web would be up for the member state to decide.

\subsubsection{eeID}

Estonia's eIDAS node access is handled by TARA \cite{tara}. 

\subsection{eID widespread adoption}

\subsubsection{eID adoption in Estonia and Lithuania}

On the surface, Estonia and Lithuania have the exact eID solutions - Bank Link, ID card, Mobile-ID, and Smart-ID. However, even with the same infrastructure, we see many inconsistencies even in the case of just these two countries.

Consider Lithuania. It is possible to connect from a centralized website \url{https://epaslaugos.lt} to access the public sector services \cite{eidasnode-lt}. Here it is possible to sign in via bank link, ID card, and Mobile-ID. Smart-ID is not part of the list. Although most banks support sign-in via three major eID providers, including Smart-ID, some listed banks like PaySera provide significant security concerns. With that bank, it is possible to access the e-government services with only email, password, and a 2FA code sent to the registered person's phone number \todo{source: I did it myself 02-27}. For this reason, Estonia's Information System Authority has taken steps to deprecate bank link \cite{ria-deprecates-bank-link} from use in TARA. In Estonia, all three major authentication options, ID card, Mobile-ID, and Smart-ID, are available to access the e-government.

\subsubsection{eIDAS notifications in Estonia and Lithuania}

For countries to communicate through the eIDAS node network, countries must notify the European Commission about what eID authentication methods they could provide \cite{eulaw-eidas}. Other countries can then use these methods to authenticate foreign citizens into their public services.

In the case of Estonia, the country has notified the European Commission about its Smart card and Mobile-ID authentication methods \cite{eulaw-eidas-notified}. Smart-ID is not a permitted method of authentication in the context of eIDAS. In Lithuania's case, only the Smart card solution is allowed - no mobile sign-in methods have been notified \cite{eulaw-eidas-notified}.

Estonia and Lithuania have shown a gap between what countries consider to be a secure and trusted source of eID and what they are willing to be held liable for in the context of eIDAS. Without a deep understanding of cultural and social intricacies, 

By having this access  This process is useful, as one service provider would be able to open up the entirety of EU market. The main issue with using eIDAS nodes as an authentication method, is the restricted access to it. In email correspondence I learnt that in Estonia, access to this service is limited to public sector only, with plans to open it up to private sector in 2022.

% ##### Attempts of eIDAS implementations in private sector

% In academic literature, there are only two well documented cases of how the private sector would access the eIDAS node network.

% ###### eID@Cloud

% The project eiD@cloud [5], conducted May 2017 to September 2018, has discovered certain issues when attempting to connect to the infrastructure. It has discovered that there's still some differences between the national schemes and the integrations of said national schemes in a unique and interoperable net that must be the eIDAS in the context of the EU, and the deployment of each eIDAS node of each member state by the national politics go at different speeds, which create mistakes and lack of availability to complete the eIDAS project. The authors are pessimistic about the prospect that fully connected Europe can be achieved soon.

% ###### LEPS

% LEPS [6], conducted September 2017 to November 2018 has tried to achieve similar goals to eID@Cloud - to identify gaps in the eIDAS infrastructure. The main challenge identified, much like in the previous research, is the lack of Service Providers, the private sector could use to interface with the eIDAS network.

% ##### eID providers in Estonia

% Applied Cyber Security Group of University of Tartu maintains a list of e-services [7] which uses at least one eID authentication method in Estonia. The following authentication methods were listed: Bank Link, ID-card, Mobile-ID, Smart-ID, TARA, and HarID. 

% ###### Bank Link

% This method of authentication is primarily created for e-services to provide close integration with banks for easier payments. Additionally, it provides an additional method of authentication to those e-services [8]. In a thesis conducted in 2012 [9], it was discovered that "Internet bank authentication [is] extremely insecure". In the years after the publication, the security got significantly better, however even then, from March of 2021 RIA no longer supports authentications with TARA, due to lack of security assessments with regards to eIDAS [https://cybersec.ee/2021/02/22/cyber-security-newsletter-2021-02-22/].

% Due to lack of security scrutiny required to satisfy eIDAS, and a poor market reach, this authentication method will not be considered  purposes of this thesis.

% ###### ID-card

% Id cards are by far the most popular way to access their eID in Estonia, which is primarily due to the legal requirement of having one. Chapter 2 of the Identity Documents Act [10] requires all EU citizens residing in Estonia to hold an ID-card, with which they could access public services online. Because of this quirk, there are more active ID-cards issued, than there are people in Estonia [11].

% There are no variable costs to allow persons to log in to websites with their ID-card, as there are no per-transaction costs for ID card authentication as the certificate validity service (OCSP) can be queried for free [12]. 

% Based on the countries internal policies, the chips on ID-cards can have different data layouts, as it is not standardized. This means that there must be a specialized piece of software to read different countries' ID-cards, requiring additional development costs.

% Certificates to be installed in ID-cards are issued by SK ID Solutions which is a trust service provider for Qualified Certificates for e-signatures [13]

% ###### Mobile-ID

% 5 years after ID cards were started to be used in Estonia, SK ID Solutions, developed a mobile phone friendly way to access the users' eID for use in Estonia, and Lithuania [14]. This was done by extending the functionality of SIM cards to make them mimic functionality of ID-cards.

% The price of using Mobile-ID for the service provider varies based on usage, staring from 10 euro per month (10ct per request), to costing over 5 000 euro, where the effective cost is under 1ct for request [15]. For the end user, mobile operators can charge extra for their monthly subscription fee, based on the contract they have with them.

% Implementation of Mobile-ID would allow service providers to access the markets of two countries: Estonia, and Lithuania, as the technical implementation is identical.

% Investigating the poor market size of users using Mobile-ID, implementation analysis will be left outside of scope of the thesis.

% Certificates to be installed in SIM cards capable of using Mobile-ID are issued by SK ID Solutions which is a trust service provider for Qualified Certificates for e-signatures [13].

% ###### Smart-ID

% Smart-ID is the latest, and fastest growing source of eID, working in all 3 of the Baltic States [16]. It utilizes mobile phones as authentication, however unlike Mobile-ID, it does not require external hardware, and everything is handled in a combination of on-server and on mobile phone. Despite that, it is still eIDAS compliant, and was recognized as a QSCD, allowing it to create QES in 2018 [17].

% The price of using Smart-ID for service providers, much like Mobile-ID varies based on usage, staring from 50 euro per month (10ct per request), to costing over 20 000 euro, where the effective cost is under 1ct for request, based on the total amount of transactions performed within a month [18]. For users, unlike Mobile-ID, there are no telecommunication operators involved and there are no costs associated to using Smart-ID.

% Implementation of Smart-ID would allow users to access the markets of three countries: Estonia, Latvia, and Lithuania.

% Part of the certificate used by Smart-ID is stored by SK ID Solutions which is a trust service provider for Qualified Certificates for e-signatures [13]. The other part is stored on the device of the user.

% ###### TARA

% TARA is the Estonia's eIDAS node interface [19]. It provides the ability for users to sign in with the 3 primary sources of identity in Estonia, and with the eID schemes of other EU member states. According to the business description [20], its use is intended for governmental agencies only, and would not work in a private setting.

% ###### eeID service

% This is a service, that uses TARA as its trust anchor for authentication, allowing users to authenticate with all of the methods provided by TARA [21]. The main difference, is that this method is targeting the private sector, enabling companies to access entirety of EU eID market.

% The service is new and does not have pricing tiers, and currently sits at 9ct per request, regardless of the amount of requests [22].

% Using eeID service would allow users to access the markets of all eIDAS countries, which will eventually include all of EU and EEA. Right now the list of countries (14) are as follows: Estonia, Germany, Italy, Spain, Belgium, Luxembourg, Croatia, Portugal, Latvia, Lithuania, Netherlands, Czech Republic, Slovakia, and Denmark.

% ###### HarID

% This service was created for the youth of Estonia to access different educational institutions across Estonia [23]. ID-cards are only legally required to be held by citizens over age of 15, so everyone under, would have unable to access their school system. HarID accepts TARA sign in methods, and username & password. This authentication method is not accessible by non-education sector, and will be skipped for the purposes of this thesis.

% ###### Dokobit

% In the initial list of services using eID in Estonia, one service stood out - Dokobit [24]. They provide resources similar to eeID, in the way that they aggregate different eID methods (ID-card, Mobile-ID, and Smart-ID), as well as eID methods from other countries. The main difference between it, and eeID, is that authentication achieved by using the native implementations of the publicly available resources of each country, rather than relying on the eIDAS infrastructure.

% Pricing for Dokobit varies drastically, and the provided prices for the Baltic States [25] starts at 50 euro per month (7.1ct per request), going down to 4.2ct per request at 500 euro per month.

% Dokobit supports 11 countries: Estonia, Italy, Spain, Belgium, Latvia, Lithuania, Finland, Norway, Iceland, Poland, and Portugal [24].

% UAB Dokobit is a trust service provider for Qualified validation of qualified e-signature. It means that the service itself does not provide certificates, but validation of signatures is considered to be trustworthy under eIDAS.

% ##### Research methodology

% ###### Development complexity

% One of primary outcomes of the research is to measure the complexity of the development. A model [26] will be used to measure the complexity of the examples and documentation provided by the services, and assign that to measurable values, which can be compared.