\section{Related Work}

This section will discuss additional literature that would provide better insight into the architectural decisions taken or the practical challenges and benefits of the integration. The results discovered by this literature will help us plan and construct the initial system architecture.

\subsection{The Austrian eID ecosystem in the public cloud: How to obtain privacy while preserving practicality}

This paper explores what information the Austrian government stores on the issued identity documents and what operations the documents can perform \cite{ZWATTENDORFER201635}. Researchers identified four types of functionality: identification and authentication of Austrian citizens, qualified electronic signature creation, encryption and decryption, and data storage. This functionality seems widely adopted as it matches Estonia's ID card.

Paper presented an interesting legal issue - Austria does not allow a person identifying code (CRR number) to be "used directly in e-Government applications due to legal data protection restrictions." The solution required Austria to create SourcePIN, a framework to develop different personal identifying numbers for each service trying to access it while hiding the original code \cite{ZWATTENDORFER201635,austria-eid-presentation}.

Authors express a big concern that everything goes through one single source of trust, which does not scale well. If many people wanted to use the system, it would quickly become a bottleneck. Moving many essential components to the public cloud can alleviate the problem.

The paper's main contribution to this thesis is to remind us that even though technological barriers are crumbling, there might still be legal obstacles to overcome. Austria is currently not part of the eIDAS node network, and it would be an excellent further research topic to investigate how Austria's eIDAS node operates.

\subsection{LEPS - Leveraging eID in the private sector}

This final research \cite{Martin2019303} was performed at a similar time to the eID@cloud \cite{guerola2019eid}, but in different countries. LEPS researchers 
have implemented an eIDAS node for private customers. However, they also provided market analysis.

The market analysis targeted four main categories of e-service providers who would be interested in integrating eID authentication:

\begin{enumerate}
    \item Organizations that need or want to migrate from the existing identity and access management (IAM) solution. This could apply to organizations that have scaled out their internal or tailor-made IAM solutions or organizations that already use partially external or third-party e-identification or authentication services but are looking for a higher level of assurance (LoA).
    \item Organizations that use low assurance third-party eID providers such as a social login want to elevate the overall level of security and decrease identity theft and fraud by integrating eIDAS eID services.
    \item Organizations that are already acting or could be acting as eID brokers.
    \item Organizations that want to open new service delivery channels through mobile phones and are interested in mobile ID solutions that work across borders.
\end{enumerate}

In the case of the thesis, the targeted e-service providers are of the first category - organizations wishing to improve IAM solutions to include a higher level of assurance.

The researchers recommend using an approach like LEPS to integrate eID authentication rather than creating an eIDAS node. The primary reason for avoiding node creation would be the cost-effectiveness of implementation. These adopters "are unlikely to have the know-how, resources, and capacity to implement eIDAS connectivity." "Many organizations do not have resources for eID service implementation and operation internally was already exploited by social networks." The targeted benefit is the "easy way to integrate highly scalable, yet low assurance, eID services."

\subsection{Federated Identity Architecture of the European eID System}

The authors of this paper describe the current situation in the identity management landscape \cite{federated-europe-identity}. The researchers provide all the necessary background information to understand the implementation details of any eID authentication system design.

\subsubsection{Authentication methods}

The first significant contribution relates to explaining different ways of authenticating persons.

Any authentication method is based on something the user knows (password, pin code, answer to security question), is (biometrics - eyes, fingerprints), or has (physical device - key card, USB device) \cite{o2003comparing}. Any other method would leave the person without agency over the authentication process.

An emphasis is put on the importance of mixing and matching these authentication schemes to increase the system's security.

\subsubsection{Authentication Paradigms and Models}

The second helpful point of the paper is the description of different identity management paradigms, and models \cite{identity-paradigms}. Paradigms refer to the implementation and deployment, whereas models refer to the data storage and roles.

The three main paradigms as network, service, and user-centric. The network-centric approach gathers all identities into one place, usually known as a "Domain Controller." The service-centric method would create a new identity for each service, leading to high duplication. The user-centric paradigm makes the user prove their own identity. Europe's eID solution does not favor any of the paradigms allowing identity providers to innovate \cite{eelaw-idcard,eeid,dokobit}.

There are also three authentication models: isolated, centralized, and federated. Unlike paradigms, where any of them is fair game, Europe's identity providers can use only the federated one. The isolated model requires all services to hold a copy of every identity in the EU. The centralized model is suitable for having a central place for looking up identity in a country, and it is an excellent solution for high-profile agencies. The federated system can scale well horizontally (adding more servers would increase the network capacity) and not require keeping an index of all citizens it would like to serve, which is a tremendous advantage when considering the GDPR requirements.

\subsubsection{Authentication protocols and services}

Researchers have allocated a good portion of the paper to provide an overview of potential protocols and implementations. The list is massive and in-depth; however, it becomes clear that SAML \cite{saml}, OAuth2.0 \cite{rfc6749}, and OpenID Connect \cite{oidc} protocols are by far the most popular protocols to choose for implementation. The engineers behind the eIDAS network implementation decided to settle on the SAML protocol.