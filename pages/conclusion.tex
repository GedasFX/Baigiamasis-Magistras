\section{Conclusion}

This thesis tried to answer the question of what is the best eID authentication solution in Estonia available to the private sector.

To answer that question, we first compiled a list of all eID solutions available to Estonia and filtered out those who offer services to the private sector. This list included {bank link}, smart cards, Mobile-ID, Smart-ID, eeID, and Dokobit. For the in-depth analysis, we chose eeID, Dokobit, and smart cards in the form of Web eID.

We investigated the architectural requirements to support eID authentication and discovered that a simple SSO server is sufficient to integrate the authentication protocol fully. However, upon further research, we found that eID authentication does not cover data confidentiality or integrity requirements if the attacker is a malicious employee. They are likely to have alternative means to access the data store, allowing them to read and tamper with data without using eID.

We compared three eID solutions: eeID, Dokobit, and Web eID. The chosen comparison criteria were trust, market reach, price, convenience, integration and maintenance costs, security, and documentation comprehensiveness.

Web eID was the most trustworthy service and was the cheapest to operate. However, it suffered from comparatively high integration and maintenance costs, was the most unfriendly solution to use, and had the lowest market reach.

Dokobit was the cheaper paid option and had the highest market reach potential, spanning multiple countries. It was also the easiest to integrate of the three. The biggest issue with this service is the matter of trust - customers may not be willing to trust a private third-party company with their personal information.

The eeID service lies somewhere between the other two solutions. It is most similar to Dokobit, but it has a poorer market reach and is more expensive. The only advantage it has over Dokobit is that people may be more willing to give their information to a government institution instead of a private company.

The documentation analysis found that Web eID had the best and most understandable documentation, followed by eeID for unclear and contradictory parts. Lastly, Dokobit did the worst by not providing any information about the hardening process.

All of the analyzed services passed the security assessment at the end of the research. Web eID and eeID did not have any issues, and Dokobit had a CSRF vulnerability on their authorization endpoint, which was disclosed and fixed.

\paragraph{Summary} The ability to integrate eID in the private sector exists; however, the public acceptance of them is limited. We were unable to find any technological limitations stopping widespread adoption. Poor market adoption is likely caused by political, economic, social, or legal reasons.