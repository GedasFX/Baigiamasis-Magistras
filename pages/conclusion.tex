\section{Conclusion}

In the thesis, we looked into the background, legality, and extensibility of eIDs, discussed the viability of using eID as a replacement for authentication, and analyzed three different eID providers the private sector could implement.

We have shown that it is possible to integrate an eID authentication scheme into a pre-existing SSO easily. The discovered challenges with integration are protecting the data from access by other means - no reason to force users to authenticate with eID if they can access the database with username and password.

We have created a privacy policy for the dummy test application considering the privacy requirements imposed by GDPR.

We have compared three different eID providers and discovered three different ways of performing cross-border authentication: integrating each provider individually, integrating a third-party provider aggregator, or tapping into a legally governed eIDAS framework.

We have interviewed with a logistics company representative to gauge the acceptance of eIDs in the general public, only to discover that it is unlikely that companies will adopt this technology without drastic changes in the market.

\paragraph{Summary} The ability to integrate eID in the private sector exists; however the public acceptance of them are limited. The technical and legal challenges associated with eID authentication make it impractical to implement. The only exception is if a company has a legitimate interest in processing personal data.