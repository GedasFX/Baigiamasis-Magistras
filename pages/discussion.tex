\section{Discussion}

This section will summarize our findings, compare the different eID solutions and discuss potential alternatives to eID authentication.

\subsection{Comparison of the eID Solutions}

We have performed an in-depth integration analysis of three eID solutions - eeID, Dokobit, and Web eID. Each of these providers has some advantage over the competition.

\subsubsection{Business Features Comparison}

Table \ref{tbl:summary-comparison-business} displays a summary of findings for each provider's business features.

\begin{table}
    \centering
    \begin{tabular}{ p{0.15\linewidth} | >{\raggedright}p{0.25\linewidth} | >{\raggedright}p{0.25\linewidth} | >{\raggedright}p{0.25\linewidth} p{0px}}
                                        & \textbf{eeID}                                         & \textbf{Dokobit}                            & \textbf{Web eID}                      & \\
        \hline
        \textbf{Trust}                  & government-backed third-party eID aggregation service & private third-party eID aggregation service & uses trust service providers directly & \\
        \hline
        \textbf{Market reach}           & multiple schemes; one country                         & multiple schemes; multiple countries        & one common scheme; multiple countries & \\
        \hline
        \textbf{Price}                  & (Most) Expensive                                      & Expensive                                   & Free                                  & \\
        \hline
        \textbf{Attack resilience}      & High                                                  & High                                        & High                                  & \\
        \hline
        \textbf{Integration complexity} & Easy                                                  & Easiest                                     & Very difficult                        & \\
        \hline
        \textbf{Platform limitations}   & None                                                  & None                                        & Requires special software             & \\
    \end{tabular}
    \caption{Business features comparison of three eID solutions: eeID, Dokobit, and Web eID}
    \label{tbl:summary-comparison-business}
\end{table}

\paragraph{eeID}

The eeID service is an excellent option if the target audience is the residents of Estonia. It is a service created and maintained by one of Estonia's governmental institutions, supports all authentication schemes used in the target demographic, and is easy to integrate. The main downside of this service is its pricing - it is by far the most expensive service of the three.

\paragraph{Dokobit}

The eeID service is an excellent option if the target audience is the same as eeID or broader in a more international sense. It is a service created and maintained by a private company in Lithuania, supports the widest variety of authentication schemes and countries, and is the most straightforward service to integrate of the three. The biggest issue for companies to adopt this service would be to accept the risk of transferring sensitive data to a third-party private company. Additionally, although it is cheaper than eeID, it would still be too expensive for most businesses.

\paragraph{Web eID}

The Web eID service is very different from the two others as it does not involve a third party, which significantly boosts security and trust. Another great advantage of this solution is that it is free to use. The main limitations of this eID solution are the comparatively weak market reach, the requirement for users to have special software installed, and the relatively astronomical integration complexity.

\subsubsection{Security and Integration Features Comparison}

This section compares the information relevant to integration - documentation quality. Table \ref{tbl:summary-comparison-docs} displays a short comparison of the three providers.

\begin{table}
    \centering
    \begin{tabular}{ >{\raggedright}p{0.25\linewidth} | >{\raggedright}p{0.2\linewidth} | >{\raggedright}p{0.2\linewidth} | >{\raggedright}p{0.2\linewidth} p{0px}}
                                                   & \textbf{eeID}                                 & \textbf{Dokobit}     & \textbf{Web eID}         & \\
        \hline
        \textbf{Integration documentation clarity} & Contradictory, no code examples               & Clear, with examples & Clear, with examples     & \\
        \hline
        \textbf{Hardening documentation clarity}   & Exists, does not explain why to do some steps & Does not exist       & Clear, with explanations & \\
    \end{tabular}
    \caption{Documentation clarity comparison of three eID solutions: eeID, Dokobit, and Web eID}
    \label{tbl:summary-comparison-docs}
\end{table}

From Table \ref{tbl:summary-comparison-docs} we see that Web eID, although being the most complicated to integrate, provided by far the best documentation of all eID solutions.

In contrast, Dokobit, while providing great integration instructions, did not give any information on how to harden the security, leaving this the responsibility of the developers to figure it out on their own.

The integration of the eeID service has left me with more questions than answers. The documentation was thorough and did cover all of the attacks, although some parts were confusing. It felt like we were doing random tasks just because a book told us to. An explanation as to why some steps are required would massively benefit the readability of the documentation.

\subsubsection{Summary}

There is no best option for choosing an eID solution provider. Ultimately, the final decision comes on the priorities of the business.

If the highest degree of trust factor is required, a company will have no other option other than to eliminate third parties, which in our case would be to use Web eID. Additionally, this option is likely most suitable for those who wish to cut costs as much as possible. For this trust service provider, the authentication operational costs are zero.

If the highest market reach is everything, adequately vetted third-party providers are the best choice. The most tangible difference between Dokobit and eeID is their market reach, and, by having a more extensive selection of supported schemes, Dokobit is the favorable option over eeID.

The only advantage eeID has over Dokobit is related to trust. It is impossible to measure trustworthiness, but if we consider that privately-held companies have more freedom over their operations over government institutions, the general public may perceive them as less trustworthy \cite{ha2004factors}. This perception could influence some companies to choose the latter option.

\subsection{System Architecture Prerequisites}

When designing the initial system, it quickly became apparent that eID is a very restricting form of authentication. To illustrate this, we have two examples.

\paragraph{Internal access} Imagine that a company allows employees to access to read data on the database. The employees can access the database after the authentication process. Usually, this access is achieved by performing Active Directory authentication or simply using a username and password.

To adequately protect the data from confidentiality breaches, the eID authentication must be performed on the database level itself, a feature that would be exceptionally difficult to integrate. As an alternative, this company would restrict access to the databases and only allows users to interface with the databases via highly audited, custom-built solutions.

This solution now encounters a different political issue - is it justifiable to invest time and effort only to introduce arbitrary hurdles for employees to do their job? There are more straightforward solutions to having stricter access control and confidentiality breaches, both legally and technologically.

\paragraph{External access}

Imagine a company that lets users open storefronts. These users can theoretically sell anything, so the government would be keen to know the sellers' identities. To avoid getting into legal troubles with the government, the website owners are enticed to enable eID authentication for putting up sales listings. This way, the storefront hosting company would have very high confidence in the user's identity.

This situation suffers from the same issue as the internal access scenario. Because the eID authentication is not performed on the database level, a rogue employee could frame someone by creating a listing for someone, as the access to the database is no longer restricted by eID.

With this limitation, eID authentication becomes useless without proof of data integrity. Digital signatures or a blockchain solution would potentially fix these cases, and further research on this topic may be beneficial.

\subsection{Alternatives to eID Authentication}

If we return to the original thesis idea of having one-time authentication, there might be an even better identity proof option for most companies.

\paragraph{The core issue with eID authentication}

In the thesis, we covered only half of the data access flow. Secure authentication methods are essential, but authorization is equally as important.

The eID authentication schemes ensure that someone is they claim they are with a high degree of certainty. This high level of assurance is not helpful if that person is not allowed to access the resources in the first place. Deciding which resources people can access requires manual role assignment by a higher authority.

Consider an example. A company is hiring a senior developer Bob. They expect a person to register soon, and they see registration at the time they would expect. What should the computer do? Should it assign the newly registered person the role of a senior? It should not, as there are no guarantees that Bob was the one who just registered - it could have been Charlie, who is not a senior-level developer. If we only look at the data provided by eID, we will see that it was, with an incredibly high degree of certainty, Charlie. If we had left the computer to assign a role automatically, we would have created an escalation of privileges vulnerability.

Alternatively, the company could assign employees to manually input the data, linking the account to a national identification code before they log in. This approach would work, but it would also defeat the whole premise of trying to skip manual verification. In short, manual confirmation is unavoidable when setting up authorization rules.

\paragraph{A more straightforward solution}

Instead of looking at electronic authentication, we can look at digital signatures instead. Consider the following flow:

\begin{enumerate}
    \item Company employee creates a quarantined account for a new employee and gives a personalized registration link.
    \item The potential client or employee visits the link and receives a file they need to sign digitally. The document's contents are not of concern; they could be a privacy policy, terms of service, or an empty text file.
    \item After the visitor signs the document, they upload it to the website. Employees can then verify that the name and surname in the digital signature match the person they were talking with online.
    \item If the link given out in step one required them to register, this account could be taken out of quarantine, as the identity was verified successfully.
\end{enumerate}

This flow fully covers the introduction's proposed use case for eIDs. This approach is not only more convenient for the user, but it is also cheaper for the company too. The only disadvantage is that it does not provide the same high assurance for the subsequent authentications as the use of an eID authentication would. However, companies can tackle this issue by hardening internal operational security policies.

Ultimately, if it is still necessary to ensure high certainty over someone's identity after each authentication, the use of eIDs is always available.