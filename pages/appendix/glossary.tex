\section*{Appendix}
\addcontentsline{toc}{section}{Appendix}

\section*{I. Glossary}
\addcontentsline{toc}{subsection}{I. Glossary}
\label{appendix:glossary}

\begin{itemize}
    \item \textbf{CA}. Certificate Authority. A certificate capable of issuing other certificates the relying parties trust.
    \item \textbf{CSRF}. Cross-Site Request Forgery. A malicious request which executes an action on behalf of the victim.
    \item \textbf{eID}. Government-issued Electronic identity.
    \item \textbf{eIDAS}. Regulation (EU) 910/2014.
    \item \textbf{GDPR}. Regulation (EU) 2016/679.
    \item \textbf{Horizontal scaling}. Expanding a system not by getting better hardware but by getting more physical machines.
    \item \textbf{HTTP}. HyperText Transfer Protocol. A common protocol used to transfer information online.
    \item \textbf{IETF}. Internet Engineering Task Force. Standards organization.
    \item \textbf{IT}. Information Technology.
    \item \textbf{LoA}. Level of Assurance. A measure of how certain a party is about something.
    \item \textbf{MFA}. Multi-factor authentication. Additional security measure users must pass before they are granted access.
    \item \textbf{OCSP}. Online Certificate Status Protocol. Used for obtaining the revocation status of an X. 509 digital certificate.
    \item \textbf{PII}. Personally identifying information.
    \item \textbf{QES}. Qualifier Electronic Signature. A signature backed by a QTSP. Legally binding signatures in the EU.
    \item \textbf{QSCD}. Qualifier Electronic Signature Creation Device. A tool capable of creating QES.
    \item \textbf{QTS}. Qualified Trust Service. A trust anchor as defined in eIDAS. Acts as a final authority for determining if a property, such as a digital signature, is authentic.
    \item \textbf{QTSP}. Qualified Trust Service Provider. A natural or legal person providing one or more QTS.
    \item \textbf{RP}. Relying Party. An entity (such as a company) that relies on a trust anchor.
    \item \textbf{SSO}. Single sign-on. A centralized authentication solution issuing access tokens for multiple services.
    \item \textbf{TLS}. Transport Layer Security. Encryption mechanism used by HTTP and other protocols.
    \item \textbf{User Agent}. An application that performs actions on behalf of a user. In the context of Web browsing, it usually is a Web Browser.
    \item \textbf{WorkAuth}. A fictitious company used in the thesis wishing to integrate the eID solutions. Used to clear up ambiguity in some places.
\end{itemize}